\documentclass{article}

\usepackage{hyperref}

\begin{document}

\title{
    Assignment 1 - Copy models for data compression \\
    \large{Algorithmic Information Theory (2022/23) \\
    Universidade de Aveiro}
}

\author{
    Martinho Tavares, 98262, martinho.tavares@ua.pt \and
    Nuno Cunha, -, - \and
    Pedro Lima, -, -
}

\date{\today}
\maketitle

% TODO: remover se tivermos referências
\nocite{*}

\section{Introduction}

\section{Work organization}

\section{Copy model}

\dots

In order to evaluate whether the copy model can provide acceptable results, we can use a baseline below which we expect the model to report the file's entropy.
We decided to use, as a baseline, the entropy considering each symbol's relative frequency in the entire file, which is given by:

$$
H(X) = - \sum_{x \in X}{p(x) \log{p(x)}}
$$

With this value in mind, we evaluated the model as a whole with different values for its parameters, on different files.
The files chosen for testing are present in the repository\footnote{\url{https://github.com/NMPC27/TAI-G7-Lab1}}, and they have the following baselines:
\begin{itemize}
    \item \verb|chry.txt|: \dots
    \item \verb|...|: \dots
\end{itemize}

\dots

Throughout this section, the different program parameters are detailed, 
and their effect on the model's performance is studied.

% \subsection{Verbose output}

\subsection{Pattern size}

When choosing a pointer in the past from which to start copying,
we need to look for an occurrence of the same $k$-sized pattern as the
one we are currently on.

Thus, $k$ is one of the parameters that affects program performance, where $k$ is a positive integer.
On one hand, a lower value of $k$

\subsubsection{Results}

\subsection{Smoothing parameter alpha}
\subsection{Base probability distribution}
\subsection{Copy pointer repositioning strategy}
\subsection{Copy pointer reposition threshold}

\section{Text generator}

\section{Conclusion}

\section{References}
\bibliography{refs}
\bibliographystyle{IEEEtran}

\end{document}


